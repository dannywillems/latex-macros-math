\usepackage{amsmath}

%-------------------------------------------------------------------------------
\newcommand{\naturel}{\mathbb{N}}
\newcommand{\integer}{\mathbb{Z}}
\newcommand{\rational}{\mathbb{Q}}
\renewcommand{\real}{\mathbb{R}} %Already defined in physics package
\newcommand{\complex}{\mathbb{C}}
%-------------------------------------------------------------------------------

%Logical
%% Basics
\def\implies{\Rightarrow}
\def\equivalence{\Leftrightarrow}
\newcommand\hyperstructure[1]{ { }^{*}#1}

%% Advanced
\def\hyperreal{\hyperstructure{\real}}
\newcommand{\hyperelementUltra}[2]{[\GSsequence{#1}{n}{\naturel}]_{#2}}
\newcommand{\hyperelement}[1]{\hyperelementUltra{#1}{\mathcal{U}}}
%-------------------------------------------------------------------------------

% Calculability
\def\recursiveFunctionsSet{\mathcal{R}}
\def\primitiveRecursiveFunctionSet{\mathcal{RP}}
\def\muRecursiveFunctionSet{\mathcal{R}_{\mu}}
%-------------------------------------------------------------------------------

%Set theory
\def\union{\cup} %Union
\def\inter{\cap} %Intersection
\newcommand{\comp}[1]{#1^{c}} %Complementary
\def\cartprod{\cross}
\newcommand{\cardinal}[1]{|#1|}
\newcommand{\powerSet}[1]{\mathcal{P}(#1)}
%-------------------------------------------------------------------------------

%Topology
\def\interior{\mathring}
\def\adh{\overline}
%-------------------------------------------------------------------------------

%Algebra

% Linear algebra

\newcommand{\matrixSpace}[2]{M_{#1}(#2)}
\newcommand{\inversibleMatrixSpace}[2]{GL_{#1}(#2)}
\newcommand{\spanspace}[1]{\left<#1\right>}

% Group theory

\def\isomorphe{\simeq}
\newcommand{\ordergroup}[1]{|#1|}
\newcommand{\GSautomorphismDef}[2]{Aut_{\text{#1}}(#2)}
\newcommand{\generatedGroup}[1]{\left<#1\right>}
\newcommand{\permutationGroup}[1]{\mathfrak{S}(#1)}

% Ring extension

\newcommand{\integralClosure}[1]{\bar{#1}}
\newcommand{\integralClosureIn}[2]{\bar{#1}_{#2}}

% Field theory

%extensionDegree
\newcommand{\extensionDegree}[2]{[#1 : #2]}
\newcommand{\extensionField}[2]{#1/#2}

%plongement
\newcommand{\plongement}[3]{Hom_{#1}(#2, #3)}

%Galois group
\newcommand{\galoisGroup}[2]{G(#1, #2)}

% Spectrum Theory
\newcommand{\spectrum}[1]{\sigma(#1)}
\newcommand{\resolvant}[1]{\rho(#1)}

\def\Ldeux{\mathcal{L}^{2}}
\def\Ldeuxstar{(\mathcal{L}^{2})^{*}}

%GSsequence :
%		#1 : represention of elements of the sequences
%		#2 : indices
%		#3 : set definition
\newcommand{\GSsequence}[3]{(#1_{#2})_{#2 \in #3}}

%GSsetDef :
%		#1 : set elements
\newcommand{\GSset}[1]{\left\{ #1 \right\}}

%GSsetDef :
%		#1 : global set
%		#2 : condition
\newcommand{\GSsetDef}[2]{\left\{#1 \, | \, #2 \right\}}

%GSprodSet :
%		#1 : indice
%		#2 : begin indice
%		#3 : end indice
%		#4 : set
\newcommand{\GSprodSet}[4]{\displaystyle \prod_{#1 = #2}^{#3} #4_{#1}}

%GSsum :
%		#1 : indice
%		#2 : begin indice
%		#3 : end indice
%		#4 : element
\newcommand{\GSsum}[4]{\displaystyle \sum_{#1 = #2}^{#3} #4}

\newcommand{\GSintervalCC}[2]{\left[#1, #2\right]}
\newcommand{\GSintervalOO}[2]{\left]#1, #2\right[}
\newcommand{\GSintervalOC}[2]{\left]#1, #2\right]}
\newcommand{\GSintervalCO}[2]{\left[#1, #2\right[}
%-------------------------------------------------------------------------------

%Analysis :

% conjuguate
\def\conjuguate{\overline}

% miLength: multi-indice length
\newcommand{\miLength}[1]{|#1|}

% segment: all points between two points given
\newcommand{\segment}[2]{S(#1, #2)}

%GSfunction :
%       #1 : name function
%       #2 : begin set
%       #3 : end set
\newcommand{\GSfunction}[3]{#1 : #2 \rightarrow #3}

%GSnorme : Deprecated --> \norm
%		#1 : elements which norme is applied on
\newcommand{\GSnorme}[1]{\norm{#1}}

%GSnormeDef :
%		#1 : elements which norme is applied on
%		#2 : norme indice
\newcommand{\GSnormeDef}[2]{\norm{#1}_{#2}}

%GSnormedSpace :
%		#1 : vectorial space
%		#2 : \GSnorme[Def] with dot as element.
\newcommand{\GSnormedSpace}[2]{(#1, #2)}

%Identification
\def\identification{\simeq}

%GSdual
%		#1 : vectorial space
\newcommand{\GSdual}[1]{#1^{*}}

%GSbidual
%		#1 : vectorial space
\newcommand{\GSbidual}[1]{#1^{**}}

\newcommand{\GSunitBoule}[1]{\mathcal{B}_{#1}}
\newcommand{\GSclosedUnitBoule}[1]{\adh{\GSunitBoule{#1}}}

\newcommand{\GSweakTopo}[1]{\sigma(#1, #1^{*})}
\newcommand{\GSpreweakTopo}[1]{\sigma(#1^{*}, #1)}

%GSendomorphism
\newcommand{\GSendomorphism}[1]{End(#1)}

%GShomomorphisme
\newcommand{\GShomomorphisme}[3][]
{
	Hom_{#1}(#2, #3)
}

%GShomomorphismeDef
% Deprecated !! Use instead directly \GShomomorsphisme
\newcommand{\GShomomorphismeDef}[3][]
{
	\GShomomorphisme{#1}{#2}{#3}
}

%GScontinueEndo
\newcommand{\GScontinueEndo}[2][]
{
	\mathcal{L}_{#1}(#2)
}

%\GScontinueHomo
\newcommand{\GScontinueHomo}[3][]
{
	\mathcal{L}_{#1}(#2; #3)
}

%\GScompactEndo
\newcommand{\GScompactEndo}[1]{\mathcal{K}(#1)}

%\GScompactHomo
\newcommand{\GScompactHomo}[2]{\mathcal{K}(#1; #2)}

\newcommand{\GSfiniteRankHomo}[2]{\mathcal{R}_{f}(#1; #2)}
\newcommand{\GSfiniteRankEndo}[1]{\mathcal{R}_{f}(#1)}

\newcommand{\GSisomorphisme}[1]{Isom(#1)}
\newcommand{\GSisomorphismeHomo}[2]{Isom(#1; #2)}
\newcommand{\GSisometryEndo}[1]{Isom(#1)}

\newcommand{\jacobienneMatrix}[2]{J_{#1}(#2)}
\newcommand{\hessienneMatrix}[2]{\mathcal{H}_{#1}(#2)}

\newcommand{\completedField}[1]{\widehat{#1}}
%-------------------------------------------------------------------------------
%Model theory

\def\satisfies{\vdash}
\newcommand{\lang}[1]{\mathcal{#1}}
\newcommand{\theory}[1]{\mathcal{#1}}
\newcommand{\struct}[1]{\mathcal{#1}}

%Definissable set
%	1 : order of the definissable sets
%	2 : the l-structure which we define the definissable sets on.
\newcommand{\definissableSet}[2][]
{
	Def^{#1}(#2)
}

%Type set
%	1 : n if it's the set of n-types.
%	2 : the theory which we build the types on.
\newcommand{\typeSet}[2]{S_{#1}(#2)}

%Ultraproduct
%	1 : indice elements
%	2 : set which contains indices
%	3 : ultrafilter
%	4 : models represention
\newcommand{\GSultraproduct}[4]{\displaystyle {\prod_{#1 \in #2}}^{#3}#4_{#1}}

%Ultrapower
%	1 : indice elements
%	2 : set which contains indices
%	3 : ultrafilter
%	4 : model
\newcommand{\ultrapower}[4]{\displaystyle {\prod_{#1 \in #2}}^{#3} #4}

%Substructures
\def\substructure{\subseteq}

%Elementary Substructures.
\def\elemSubstructure{\preceq}

\def\existentiallyClosed{\underset{e.c}{\subseteq}}
%-------------------------------------------------------------------------------

%	Hilbert space
\def\Hilbert{\mathcal{H}}
\newcommand{\GSortho}[1]{#1^{\perp}}
\def\GSid{\cong}
\newcommand{\dotprod}[2]{\bra{#1}\ket{#2}}
\newcommand{\adjointe}[1]{#1^{*}}
%-------------------------------------------------------------------------------

%	Group representions
\newcommand{\GSrepr}[2]{Repr(#1, #2)}
\newcommand{\GSreprf}[2]{Repr_{f}(#1, #2)}
\newcommand{\GSrepri}[2]{Repr_{i}(#1, #2)}
%-------------------------------------------------------------------------------

%Probability

%	Borelian
\newcommand{\borelian}[1]{\mathcal{B}(#1)}

%	Laws
\newcommand{\lawBernouilli}[1]{\mathcal{B}(1, #1)}
\newcommand{\lawBinomial}[2]{\mathcal{B}(#1, #2)}
\newcommand{\lawExponential}[1]{e(#1)}
\newcommand{\lawPoisson}[1]{\mathcal{P}(#1)}
\newcommand{\lawNormal}[2]{\mathcal{N}(#1, #2)}
\newcommand{\lawUniform}[1]{\mathcal{U}(#1)}
\newcommand{\lawCauchy}{\mathcal{C}}

%	Variables
\def\varFollow{\sim}
\def\varSameLaw{\overset{\mathcal{L}}{=}}
\def\varIndependant{\perp}
%-------------------------------------------------------------------------------
